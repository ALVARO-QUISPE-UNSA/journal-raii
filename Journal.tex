\documentclass[journal]{IEEEtran}
\usepackage[utf8]{inputenc} %para más paquetes
\usepackage{cite}
\usepackage{amsmath,amsfonts}
\usepackage{algorithmic}
\usepackage{array}
\usepackage[caption=false,font=normalsize,labelfont=sf,textfont=sf]{subfig}
\usepackage{textcomp}
\usepackage{stfloats}
\usepackage{url}
\usepackage{verbatim}
\usepackage{graphicx}
\hyphenation{op-tical net-works semi-conduc-tor IEEE-Xplore}
\def\BibTeX{{\rm B\kern-.05em{\sc i\kern-.025em b}\kern-.08em
    T\kern-.1667em\lower.7ex\hbox{E}\kern-.125emX}}
\usepackage{balance}
\usepackage[spanish]{babel}
%\setlength{\parskip}{\baselineskip} % Espacio entre párrafos

%\renewcommand{\thesection}{}
% para que babel no afecte las subsecciones
\renewcommand{\thesubsection}{\Alph{subsection}}
%\renewcommand{\thesection}{\Alph{section}}




%en docs:
% https://docs.google.com/document/d/1wWdCUyAu67M-DCEVUwaJbqJ3g066-fcK/edit?usp=sharing&ouid=106774337818310092672&rtpof=true&sd=true
% COMIENZO DEL DOCUMENTO
\begin{document}

\title{Análisis Comparativo del uso de EvolvRehab para la mejora de miembros inferiores en pacientes con ACV: 4 casos}

\author{%
\IEEEauthorblockN{Alvaro Raúl Quispe Condori}\\
\IEEEauthorblockA{\textit{Universidad Nacional de San Agustín}, Arequipa, Perú\\
aquispecondo@unsa.edu.pe}
}
\author{Alvaro Raúl Quispe Condori\\
\IEEEauthorblockA{\textit{Universidad Nacional de San Agustín de Arequipa}, Perú\\
aquispecondo@unsa.edu.pe}
}
%\thanks{Hago mis agradecimientos}}

%ocultar el encabezado, aunque lo vea bonito
%\markboth{Journal of \LaTeX\ Class Files,~Vol.~18, No.~9, September~2020}%
%{How to Use the IEEEtran \LaTeX \ Templates}

\maketitle

\begin{abstract}

\end{abstract}

\begin{IEEEkeywords}
    EvolvRehab, ACV, rehabilitación virtual.
\end{IEEEkeywords}


\section{Introducción}
\section{Rehabilitación virtual}
En la presente sección se expondrá la definición de la rehabilitación virtual y los beneficios que posee frente a la terapia tradicional.
\subsection{Definición}
Se puede hablar de la rehabilitación virtual como un ambiente simulado \cite{sutherland1965ultimate} que utiliza los más recientes avances tecnológicos con el fin de establecer metas y retroalimentación de alta calidad a los pacientes que poseen algún tipo de discapacidad. Resultados que serían inalcanzables con la terapia tradicional \cite{weiss2009moving}.

Así mismo, Levin \cite{levin2020potential} habla de la rehabilitación virtual como el arte y la ciencia del desarrollo y la aplicación de actividades cognitivas y/o físicas interactivas. Las cuales están modeladas a través de computadoras, que simulan en su aspecto y sensación a objetos y acciones del mundo real. 

En consecuencia, podemos interpretar la rehabilitación virtual como el empleo de tecnología (computadores o sensores) con el fin de obtener información de los pacientes, la cual permita establecer una evaluación eficaz con objetivos oportunos.


\subsection{Beneficios}
Uno de los beneficios de la rehabilitación virtual es el nivel de desafío y retroalimentación que proporciona a los pacientes \cite{levin2020potential}. La cual contribuye a desarrollar algunos de los principios de la plasticidad basada en la experiencia \cite{kleim2008principles}, como la especificidad o la intensidad.

Por otro lado, el amplio abanico de ejercicios e intensidades que proporciona la rehabilitación virtual, promueve la adaptabilidad y el aprendizaje para encontrar soluciones a los diferentes problemas motores \cite{levin2020potential}. De esta manera, la rehabilitación virtual se diferencia de los enfoques de entrenamiento repetitivo, los cuales no generarían un cambio significativo en la corteza motora \cite{levin2020potential, nudo2013recovery}.

Por último, este estilo de rehabilitación ayuda a los pacientes a enfrentar el estrés y la ansiedad \cite{dolatabadi2017toronto}. Factor que convierte a la rehabilitación en una actividad atractiva, que genera el deseo del individuo a practicar durante más tiempo, promoviendo la plasticidad neuronal \cite{levin2020potential}. 

\section{EvolvRehab}
En esta sección se detallará la definición de EvolvRehab y la tecnología que emplea en sus diferentes productos.  

\subsection{Definición}
%que es, objetivo, características, 
Según su sitio web oficial \cite{evolvRehab}, EvolvRehab es una plataforma de rehabilitación creada por el fabricante de dispositivos médicos Evol. Según la empresa, sus dispositivos fueron diseñados con el objetivo de mejorar el proceso de rehabilitación, abordando impedimentos como el acceso a la rehabilitación y la adherencia a la terapia.

Por otro lado, autores como Ellis et al. \cite{ellis2022consideration} describen a EvolvRehab como un sistema de rehabilitación virtual no inmersivo, puesto que no requiere de dispositivos como gafas de realidad aumentada. De esta manera, EvolvRehab se vuelve una alternativa accesible para obtener altas dosis de ejercicios basados en la rehabilitación de accidentes cerebrovasculares \cite{ellis2022consideration, evolvRehab}.

\subsection{Tecnología empleada}
%tipos de productos de evolvRehab, impedimentos que aborda, productos
Los productos que forman parte de EvolvRehab incluyen hardware y software especializado, debido a que sus beneficios son mayores con respecto a los dispositivos y sensores dirigidos a un público general \cite{maier2019effect}. Así, la amplia variedad de productos que incluye EvolvRehab varía según el producto y la parte del cuerpo objetiva.

Podemos mencionar algunos kits como EvolvRehab-Hands, que usan sensores de Leap Motion para entrenar la destreza manual fina \cite{evolvRehab}, u otros productos que trabajan con Microsoft Kinect, como EvolvRehab-Body, los cuales están orientados a la terapia de los miembros superiores \cite{ellis2022consideration}.



\section{Metodología}
\subsection{Caso 1}
Institución y lugar: Según 

Participantes: cantidad de participantes (Grupo de control y grupo experimental), edades, enfermedad, genero.

Tecnología utilizada: software/hardware 

Protocolo o intervención: cuantas sesiones, duración de sesiones, cuantas semanas o meses?

Instrumentos: como se evaluó el resultado: test clínicos o test de ingeniería 

\subsection{Caso 2}
Institución y lugar

Participantes: cantidad de participantes (Grupo de control y grupo experimental), edades, enfermedad, genero.

Tecnología utilizada: software/hardware 

Protocolo o intervención: cuantas sesiones, duración de sesiones, cuantas semanas o meses?

Instrumentos: como se evaluó el resultado: test clínicos o test de ingeniería 


\subsection{Caso 3}
Institución y lugar

Participantes: cantidad de participantes (Grupo de control y grupo experimental), edades, enfermedad, genero.

Tecnología utilizada: software/hardware 

Protocolo o intervención: cuantas sesiones, duración de sesiones, cuantas semanas o meses?

Instrumentos: como se evaluó el resultado: test clínicos o test de ingeniería 


\subsection{Caso 4}
Contexto (institución y lugar), tiempo que duró la experiencia, cantidad de participantes, resultados obtenidos, software/hardware utilizado, metodología utilizados (cuantas sesiones, duración de sesiones, cuantas semanas o meses) 

\section{Resultados y Discución}
\subsection{Diferencias}
\subsection{Discución}
\subsection{Propuesta}
\subsection{Semejanzas}






% lo que hace es tratar de igualar el final de cada página 
\balance

\bibliographystyle{IEEEtran}
\bibliography{referencias}



% para briografía
%\begin{IEEEbiographynophoto}{Jane Doe}
%Biography text here without a photo.
%\end{IEEEbiographynophoto}


\end{document}


