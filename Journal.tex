\documentclass[journal]{IEEEtran}
\usepackage[utf8]{inputenc} %para más paquetes
\usepackage{cite}
\usepackage{amsmath,amsfonts}
\usepackage{algorithmic}
\usepackage{array}
\usepackage[caption=false,font=normalsize,labelfont=sf,textfont=sf]{subfig}
\usepackage{textcomp}
\usepackage{stfloats}
\usepackage{url}
\usepackage{verbatim}
\usepackage{graphicx}
\hyphenation{op-tical net-works semi-conduc-tor IEEE-Xplore}
\def\BibTeX{{\rm B\kern-.05em{\sc i\kern-.025em b}\kern-.08em
    T\kern-.1667em\lower.7ex\hbox{E}\kern-.125emX}}
\usepackage{balance}
\usepackage[spanish]{babel}
\setlength{\parskip}{\baselineskip} % Espacio entre párrafos

%\renewcommand{\thesection}{}
% para que babel no afecte las subsecciones
\renewcommand{\thesubsection}{\Alph{subsection}}
%\renewcommand{\thesection}{\Alph{section}}




%en docs:
% https://docs.google.com/document/d/1wWdCUyAu67M-DCEVUwaJbqJ3g066-fcK/edit?usp=sharing&ouid=106774337818310092672&rtpof=true&sd=true
% COMIENZO DEL DOCUMENTO
\begin{document}

\title{Análisis Comparativo del uso de EvolvRehab para la mejora de miembros inferiores en pacientes con ACV: 4 casos}

\author{%
\IEEEauthorblockN{Alvaro Raúl Quispe Condori}\\
\IEEEauthorblockA{\textit{Universidad Nacional de San Agustín}, Arequipa, Perú\\
aquispecondo@unsa.edu.pe}
}
\author{Alvaro Raúl Quispe Condori\\
\IEEEauthorblockA{\textit{Universidad Nacional de San Agustín de Arequipa}, Perú\\
aquispecondo@unsa.edu.pe}
}
%\thanks{Hago mis agradecimientos}}

%ocultar el encabezado, aunque lo vea bonito
%\markboth{Journal of \LaTeX\ Class Files,~Vol.~18, No.~9, September~2020}%
%{How to Use the IEEEtran \LaTeX \ Templates}

\maketitle

\begin{abstract}
   
   
   
   
\end{abstract}

\begin{IEEEkeywords}
    EvolvRehab, ACV, rehabilitación virtual.
\end{IEEEkeywords}


\section{Introducción}
\section{KEYWORD 2 (MARCO TEÓRICO)}
\subsection{Sustema 1}
\subsection{Sustema 2}
\subsection{Sustema 3}

\section{EvolvRehab}
A continuación se detallará el concepto de EvolvRehab, incluyendo sus características y productos incluidos. Al igual que su enfoque en la rehabilitación virtual junto a los impedimentos que promete afrontar. 
\subsection{Definición y características}
%que es, objetivo,  características, 
Según su sitio web oficial \cite{evolvRehab}, EvolvRehab es una plataforma de rehabilitación creada por el fabricante de dispositivos médicos Evol. Según la empresa, sus dispositivos fueron diseñados con el objetivo de mejorar el proceso de rehabilitación, abordando impedimentos como el acceso a la rehabilitación y la adherencia a la terapia.

Por otro lado, autores como Ellis et al. \cite{ellis2022consideration} describen a EvolvRehab como un sistema de rehabilitación virtual no inmersivo, puesto que no requiere de dispositivos como gafas de realidad aumentada. De esta manera, EvolvRehab se vuelve una alternativa accesible para obtener altas dosis de ejercicios basados en la rehabilitación de accidentes cerebrovasculares \cite{ellis2022consideration, evolvRehab}.

\subsection{Tecnología empleada}
%tipos de productos de evolvRehab, impedimentos que aborda, productos
Los productos que forman parte de EvolvRehab incluyen hardware y software especializado, debido a que sus beneficios son mayores con respecto a los dispositivos y sensores dirigidos a un público general \cite{maier2019effect}. Así, la amplia variedad de productos que incluye EvolvRehab varía según el producto y la parte del cuerpo objetiva.

Podemos mencionar algunos kits como EvolvRehab-Hands, que usan sensores de Leap Motion para entrenar la destreza manual fina \cite{evolvRehab}, u otros productos que trabajan con Microsoft Kinect, como EvolvRehab-Body, los cuales están orientados a la terapia de los miembros superiores \cite{ellis2022consideration}.
\subsection{Sustema 3}



\section{Metodología}
\subsection{Caso 1}
Institución y lugar: Según 

Participantes: cantidad de participantes (Grupo de control y grupo experimental), edades, enfermedad, genero.

Tecnología utilizada: software/hardware 

Protocolo o intervención: cuantas sesiones, duración de sesiones, cuantas semanas o meses?

Instrumentos: como se evaluó el resultado: test clínicos o test de ingeniería 

\subsection{Caso 2}
Institución y lugar

Participantes: cantidad de participantes (Grupo de control y grupo experimental), edades, enfermedad, genero.

Tecnología utilizada: software/hardware 

Protocolo o intervención: cuantas sesiones, duración de sesiones, cuantas semanas o meses?

Instrumentos: como se evaluó el resultado: test clínicos o test de ingeniería 


\subsection{Caso 3}
Institución y lugar

Participantes: cantidad de participantes (Grupo de control y grupo experimental), edades, enfermedad, genero.

Tecnología utilizada: software/hardware 

Protocolo o intervención: cuantas sesiones, duración de sesiones, cuantas semanas o meses?

Instrumentos: como se evaluó el resultado: test clínicos o test de ingeniería 


\subsection{Caso 4}
Contexto (institución y lugar), tiempo que duró la experiencia, cantidad de participantes, resultados obtenidos, software/hardware utilizado, metodología utilizados (cuantas sesiones, duración de sesiones, cuantas semanas o meses) 

\section{Resultados y Discución}
\subsection{Diferencias}
\subsection{Discución}
\subsection{Propuesta}
\subsection{Semejanzas}






% lo que hace es tratar de igualar el final de cada página 
\balance

\bibliographystyle{IEEEtran}
\bibliography{referencias}



% para briografía
%\begin{IEEEbiographynophoto}{Jane Doe}
%Biography text here without a photo.
%\end{IEEEbiographynophoto}


\end{document}


